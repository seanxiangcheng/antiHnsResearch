\documentstyle[11pt,apsab]{article}
\nofiles
\MeetingID{MAR15}
%\DateSubmitted{20141111}
\LogNumber{MAR15-2014-001631}
\SubmittingMemberSurname{Cheng}
\SubmittingMemberGivenName{Xiang}
%\SubmittingMemberID{61119255,USA}
\SubmittingMemberEmail{xiang.cheng@emory.edu}
\SubmittingMemberAffil{Department 
of Physics, Emory 
University}
\PresentationType{oral}
\SortCategory{2.6}{}{}{}
\received{11 Nov 
2014}
\begin{document}
\Title{Antiferromagnetic Ising Model in Hierarchical 
Networks}
\titlenote{Supported through NSF grant 
DMR-1207431}
\AuthorSurname{Cheng}
\AuthorGivenName{Xiang}
%\AuthorEmail{xiang.cheng@emory.edu}
\AuthorSurname{Boettcher}
\AuthorGivenName{Stefan}
%\AuthorEmail{sboettc@emory.edu}
\AuthorAffil{Department 
of Physics, Emory University}
\CategoryType{C}
\begin{abstract}
The Ising 
antiferromagnet  is a convenient model of glassy dynamics.  It can introduce 
geometric frustrations and may give rise to a spin glass phase and glassy 
relaxation at low temperatures $[1]$.  We apply the antiferromagnetic Ising 
model to 3 hierarchical networks which share features of both small world 
networks and regular lattices. Their recursive and fixed structures make them 
suitable for exact renormalization group analysis as well as numerical 
simulations. We first explore the dynamical behaviors using simulated annealing 
and discover an extremely slow relaxation at low temperatures. Then we employ 
the Wang-Landau algorithm to investigate the energy landscape and the 
corresponding equilibrium behaviors for different system sizes. Besides the 
Monte Carlo methods, renormalization group $[2]$ is used to study the 
equilibrium properties in the thermodynamic limit and to compare with the 
results from simulated annealing and Wang-Landau sampling.  \\ 
$[1]$ C. P. Herrero, Phys. Rev. E. {\bf 77}, 04112 (2008)\\ 
$[2]$ V. Singh, C. T. Brunson, S. Boettcher, arXiv:1408.0669 
(2014)
\end{abstract}
\end{document}