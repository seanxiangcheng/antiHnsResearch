\documentstyle[11pt,apsab]{article}

\nofiles
\MeetingID{MAR16}
%\DateSubmitted{20141111}
\SubmittingMemberSurname{Cheng}
\SubmittingMemberGivenName{Xiang}
%\SubmittingMemberID{61119255,USA}
\SubmittingMemberEmail{xiang.cheng@emory.edu}
\SubmittingMemberAffil{Department of Physics, Emory University}
\PresentationType{oral}
\SortCategory{2.6}{}{}{}
\received{11 Nov 2015}
\begin{document}
\Title{Aging in the two-dimensional random-field systems}
\titlenote{Supported through NSF grant DMR-1207431}
\AuthorSurname{Cheng}
\AuthorGivenName{Xiang}
%\AuthorEmail{xiang.cheng@emory.edu}
\AuthorSurname{Boettcher}
\AuthorGivenName{Stefan}
\AuthorSurname{Urazhdin}
\AuthorGivenName{Sergei}
\AuthorSurname{Ma}
\AuthorGivenName{Tianyu}
%\AuthorEmail{sboettc@emory.edu}
\AuthorAffil{Department of Physics, Emory University}
\CategoryType{C}
\begin{abstract}
Random fields introduced into the classical Ising and Heisenberg spin models can roughen the energy landscape, leading to complex nonequilibrium dynamics. The effects of random fields on magnetism have been previously studied in the context of dilute antiferromagnets (AF), impure substrates, and magnetic alloys $[1]$. We utilized random-field spin models to simulate the observed magnetic aging in thin-film ferromagnet/antiferromagnet (F/AF) bilayers. Our experiments show extremely slow cooperative relaxation over a wide range of temperatures and magnetic fields $[2]$. In our simulations, the experimental system is coarse-grained into a random field Ising model on a 2D square lattice. Monte Carlo simulations indicate that aging processes may be associated with the glassy evolution of the magnetic domain walls, due to the pinning by the random fields. The scaling of the simulated aging agrees well with experiments. Both are consistent with either a small power-law or logarithmic dependence on time. We further discuss the topological effects on aging due to the dimensional crossover from the Ising to the Heisenberg regime.\\
$[1]$T. Nattermann, Spin glasses and random fields, 12 (1997):277\\
$[2]$ S. Urazhdin, arXiv:1503.08380 (2015)(arxiv.org/pdf/1503.08380.pdf)

\end{abstract}
\end{document}